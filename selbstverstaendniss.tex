\documentclass[10pt,a4paper]{article}
\usepackage{ucs}
\usepackage{eurosym}
\usepackage[utf8x]{inputenc}
\usepackage[ngerman]{babel}
\usepackage[margin=3cm]{geometry}
\newcommand{\qt}[1]{\glq\emph{#1}\grq}
\newcommand{\qs}[1]{\glqq#1\grqq}
\newcommand{\name}{FOOSPACE}
\newcommand{\revision}{$Revision: 1$}
\setlength{\parskip}{6pt}
\setlength{\parindent}{0pt}
\usepackage{enumerate}
\usepackage{color}
\pagestyle{plain}
\usepackage{palatino}
\usepackage[bookmarks,bookmarksopen=true,bookmarksnumbered=true,colorlinks,linkcolor=black,urlcolor=blue]{hyperref}
\usepackage{ulem}
\begin{document}

\section*{Selbstverständniss}

Der FOOSPACE setzt es sich zum Ziel eine freundliche und produktive Umgebung
für technikinteressierte Menschen zu schaffen.  Wir glauben nicht, dass es
hilfreich ist, hier eine formale Bestimmung für jenen Inhalt zu geben welchen
wir mit dem bekannten Satz vom "`Spass am Gerät"' zusammengefasst wissen wollen.
Wir verstehen und begrüssen, dass Technologie nie nur im luftleeren Raum
existiert und sind offen für all die verschiedenen Formen und Überschneidungen
an denen sich Technik, Politik, Kunst und Kultur treffen.
Diese Offenheit soll jedoch nicht den Blick darauf verstellen, dass der FOOSPACE in
erster Linie ein 'Hackerspace' ist welcher sich primär durch eben
jenen Technologiebezug definiert.  Damit wollen wir ausdrücklich nicht sagen, dass
Politik, Kunst, Kultur, alternative politische Entwürfe und all die damit
einhergehenden Gruppen und sozialen Gefüge irrelevant sind. Was wir sagen ist:
sie sind nicht der Kern dessen was wir versuchen zu tun.
\end{document}
