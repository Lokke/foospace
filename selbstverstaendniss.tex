\documentclass[10pt,a4paper]{article}
\usepackage{ucs}
\usepackage{eurosym}
\usepackage[utf8x]{inputenc}
\usepackage[ngerman]{babel}
\usepackage[margin=3cm]{geometry}
\newcommand{\qt}[1]{\glq\emph{#1}\grq}
\newcommand{\qs}[1]{\glqq#1\grqq}
\newcommand{\name}{dezentrale}
\newcommand{\revision}{$Revision: 1$}
\newcommand{\documentstatus}{Entwurf (DRAFT)}
\setlength{\parskip}{6pt}
\setlength{\parindent}{0pt}
\usepackage{enumerate}
\usepackage{color}
\pagestyle{plain}
\usepackage{palatino}
\usepackage[bookmarks,bookmarksopen=true,bookmarksnumbered=true,colorlinks,linkcolor=black,urlcolor=blue]{hyperref}
\usepackage{ulem}
\begin{document}
\title{Selbstverst{\"a}ndniss}

\section*{Dokumentenstatus}
\documentstatus

\section*{Selbstverst{\"a}ndniss}

Der \name\ e.V. setzt es sich zum Ziel eine freundliche und produktive Umgebung
f{\"u}r technikinteressierte Menschen zu schaffen.  Wir glauben nicht das es
hilfreich ist hier eine formale Bestimmung f{\"u}r jenen Inhalt zu geben welchen
wir mit dem bekannten Satz vom \qs{Spass am Ger{\"a}t} zusammengefasst wissen wollen.
Wir verstehen und begr{\"u}ssen das Technologie nie nur im luftleeren Raum
existiert und sind offen f{\"u}r all die verschiedenen Formen und {\"U}berschneidungen
an denen Technik Politik, Kunst und Kultur trifft.
Diese Offenheit soll jedoch den Blick darauf verstellen das der FOOSPACE in
erster Linie ein \qs{Hackerspace} ist welcher sich prim{\"a}r durch eben
jenen Technologiebezug bestimmt.  Damit wollen wir ausdr{\"u}cklich nicht sagen das
Politik, Kunst, Kultur, alternative politische Entw{\"u}rfe und all die damit
einhergehenden Gruppen und sozialen Gef{\"u}ge irrelevant sind! Was wir sagen ist:
sie sind nicht der Kern dessen was wir versuchen zu tun.
\end{document}
