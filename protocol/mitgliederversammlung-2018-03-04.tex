\documentclass[10pt,a4paper]{scrartcl}
\usepackage{eurosym}
\usepackage[margin=3cm]{geometry}
\usepackage[utf8]{inputenc}
\newcommand{\qt}[1]{\glq\emph{#1}\grq}
\newcommand{\qs}[1]{"#1"}
\newcommand{\name}{dezentrale}
\newcommand{\revision}{$Revision: 2018-03-04$}
\newcommand{\eventdate}{04.03.2018}
\newcommand{\schriftfuehrer}{Jan Hollburg}
\newcommand{\documentstatus}{Ver{\"o}ffentlicht (PUBLIC)}
\setlength{\parskip}{6pt}
\setlength{\parindent}{0pt}
\usepackage{enumerate}
\usepackage{color}
\pagestyle{plain}
\usepackage{palatino}
\usepackage[bookmarks,bookmarksopen=true,bookmarksnumbered=true,colorlinks,linkcolor=black,urlcolor=blue]{hyperref}
\begin{document}
\title{Protokoll - MITGLIEDERVERSAMMLUNG \qs{\name\ e.V.}}
{\LARGE Protokoll - MITGLIEDERVERSAMMLUNG\\ \qs{\name\ e.V.}}

\section*{Dokumentenstatus}
\documentstatus\\
Fassung vom \eventdate\ (\revision)

\section*{{\"U}berblick}
    Ort: dezentrale e.V., Dreilindenstr. 19, 04177 Leipzig\\
    Datum: \eventdate

\section*{Tagesordnung}
	Die folgenden Punkte stehen auf der Tagesordnung der Versammlung:
	\begin{enumerate}
		\item Bestimmung des Versammlungsleiters, sowie Protokollanten
		\item Feststellung der ordnungsgem{\"a}{\ss}en Einberufung
		\item Feststellung der Beschlussf{\"a}higkeit
		\item Genehmigung der Tagesordnung
		\item Genehmigung des Protokolls der Gr{\"u}ndungsversammlung
		\item Bericht(e) des Vorstands
		\item Abstimmung {\"u}ber Vorschlag zur Satzungs{\"a}nderung
		\item Abstimmung {\"u}ber Vorschlag zur Gesch{\"a}ftsordnungs{\"a}nderung
		\item Verschiedenes
	\end{enumerate}

\section*{1) Bestimmung des Versammlungsleiters, sowie Protokollanten}
	Versammlungsleiter: Manuel Madrenes (Vorstandsvorsitzender)\\
	Protokollant: Jan Hollburg (Schriftf{\"u}hrer)

\section*{2) Feststellung der ordnungsgem{\"a}{\ss}en Einberufung}
	Die Einladung zur MV mit o.g. Tagesordnung erfolgte am 17.02.2018 an members@dezentrale.space.\\
	Die MV wurde ordnungsgem{\"a}{\ss} einberufen.

\section*{3) Feststellung der Beschlussf{\"a}higkeit}
	Der dezentrale e.V. hat aktuell 18 Mitglieder, davon 17 regul{\"a}r und ein F{\"o}rdermitglied.\\
	Laut Gesch{\"a}ftsordnung sind MV beschlussf{\"a}hig ab 51\% der regul{\"a}ren Mitglieder,
	das sind zum Termin der MV 9 von 17 regul{\"a}ren Mitgliedern.\\
	Anwesenheit 15:30 Uhr: 8 regul{\"a}re Mitglieder. Die MV ist damit nicht beschlussf{\"a}hig.\\
\\
	Es sind 8 Mitglieder anwesend. Die MV wird daher ohne Beschlussf{\"a}higkeit durchgef{\"u}hrt.\\
\\
	Die Namen der teilnehmenden Mitglieder wurden protokolliert, nach GO §4 Punkt 3.

\section*{4) Genehmigung der Tagesordnung}
	Mangels Beschlussf{\"a}higkeit wird dieser Punkt {\"u}bersprungen.

\section*{5) Genehmigung des Protokolls der Gr{\"u}ndungsversammlung}
	Mangels Beschlussf{\"a}higkeit wird dieser Punkt {\"u}bersprungen.

\section*{6) Bericht(e) des Vorstands}
	\begin{itemize}
		\item 2 Vorstandssitzungen f{\"u}hrten zu Satzungs-/GO-{\"A}nderungen:
		\begin{itemize}
			\item 2017-06-21 {\"A}nderungen nach Vorgabe des Amtsgerichtes
			\item 2017-07-02 {\"A}nderungen nach Vorgabe des Finanzamtes
			\item Die {\"A}nderungen wurden in der MV erl{\"a}utert (Siehe Satzung §9 Punkt 3).
		\end{itemize}
	
		\item Es wurden Veranstaltungen durchgef{\"u}hrt und mitgestaltet:
		\begin{itemize}
			\item Er{\"o}ffnungsveranstaltung des Vereins am 04.11.2017
			\item 2 Adventshacks
			\item Besuch des Shackspace in Stuttgart, zur Vernetzung mit anderen Hackerspaces
			\item Besuch und Mitgestaltung des Chaos Communication Congress 2017 (34C3)
			\item Bereitstellung der R{\"a}ume f{\"u}r das FreeCodeCamp regelm{\"a}{\ss}ig Montags, seit 2017-10
			\item Hardware-Bastelrunde regelm{\"a}{\ss}ig Dienstags, seit 2017-09
			\item Bitcoin-Themenabend regelm{\"a}{\ss}ig am ersten Mittwoch des Monats, seit 2017-11
			\item Techniksprechstunde regelm{\"a}{\ss}ig Donnerstags, seit 2017-10
			\item Computerspieleabend regelm{\"a}{\ss}ig Freitags, seit 2017-10
		\end{itemize}

		\item Infrastruktur f{\"u}r den Verein wurde aufgebaut:
		\begin{itemize}
			\item Einrichtung einer K{\"u}che
			\item Vernetzung ins Vorderhaus (Glasfaster, Ethernet, Koaxial-Antennenkabel)
			\item schnelles Internet
			\item T{\"u}rlogo
			\item Elektronisches {\"O}ffnen der (Stra{\ss}en-)t{\"u}r
		\end{itemize}

		\item Zukunft:
		\begin{itemize}
			\item Weiterf{\"u}hren obiger Projekte
			\item Soziale Vernetzung mit anderen Hackerspaces und Vereinen
			\item L{\"o}ttischworkshops, L{\"o}ttischeinweihungsparty
			\item Chaos Communication Congress
			\item Die angefangene Entwicklung der Webseite fortf{\"u}hren,
			           um relevantere Informationen zu pr{\"a}sentieren
		\end{itemize}

		\item Finanzen:
		\begin{itemize}
			\item Konto: 90,76 EUR (Stand: 2018-03-04)
			\item Ausst{\"a}nde:
			\begin{itemize}
				\item Stromrechnung f{\"u}r Holzwerkstatt 360 EUR
				\item 4 Mitglieder haben noch nicht alle Mitgliedsbeitr{\"a}ge gezahlt
			\end{itemize}
			\item Verbindlichkeiten: keine.
			\item Verm{\"o}gen:
			\begin{itemize}
				\item Keine nennenswerten beweglichen G{\"u}ter
				\item Kaution f{\"u}r die Vereinsr{\"a}ume: 750 EUR
			\end{itemize}
			\item Ausgaben {\"u}ber 100 Euro:
			\begin{itemize}
				\item Miete: 450 Eur pro Monat warm
				\item Strom: 70 Eur pro Monat Abschlag
				\item Internet: 14,99 pro Monat (erste 12 Monate)
			\end{itemize}
			\item Barspenden:
			\begin{itemize}
				\item Techniksprechstunde: 512 Euro
				\item einzelnes Mitglied: 410 Euro
				\item Quartiermeister Projektf{\"o}rderung: 400 Euro
				\item Spendenbox (August 2017 bis heutiges Datum): 117,75 Euro
			\end{itemize}
			\item Ausgaben monatlich: 550 Euro
			\item Einnahmen:
			\begin{itemize}
				\item August bis November: regelm{\"a}{\ss}ig ein paar neue Mitglieder; seit Dezember eher Flaute
				\item Oktober: 2 neue Mitglieder
				\item November: 2 neue Mitglieder
				\item Februar: 2 neue Mitglieder
			\end{itemize}
		\end{itemize}
	\end{itemize}

\section*{7) Abstimmung {\"u}ber Vorschlag zur Satzungs{\"a}nderung}
	Mangels Beschlussf{\"a}higkeit wird dieser Punkt {\"u}bersprungen.

\section*{8) Abstimmung {\"u}ber Vorschlag zur Gesch{\"a}ftsordnungs{\"a}nderung}
	Mangels Beschlussf{\"a}higkeit wird dieser Punkt {\"u}bersprungen.

\section*{9) Verschiedenes}
	Die Mitgliederversammlung wurde ohne Abstimmungen beendet. Es soll ein
	zeitnaher Termin f{\"u}r eine weitere Versammlung gefunden werden.
\\
\\
\\
\\
Leipzig, den \eventdate \ \ \ Protokollant: \schriftfuehrer
\end{document}
