\documentclass[10pt,a4paper]{scrartcl}
\usepackage{eurosym}
\usepackage[utf8x]{inputenc}
\usepackage{ngerman}
\usepackage[ngerman]{babel}
\usepackage[margin=3cm]{geometry}
\newcommand{\qt}[1]{\glq\emph{#1}\grq}
\newcommand{\qs}[1]{\glqq#1\grqq}
\newcommand{\name}{dezentrale}
\newcommand{\revision}{$Revision: 2017-07-23$}
\newcommand{\eventdate}{23.07.2017}
\newcommand{\schriftfuehrer}{Jan Hollburg}
\newcommand{\documentstatus}{Ver{\"o}ffentlicht (PUBLIC)}
\setlength{\parskip}{6pt}
\setlength{\parindent}{0pt}
\usepackage{enumerate}
\usepackage{color}
\pagestyle{plain}
\usepackage{palatino}
\usepackage[bookmarks,bookmarksopen=true,bookmarksnumbered=true,colorlinks,linkcolor=black,urlcolor=blue]{hyperref}
\begin{document}
\title{Protokoll - Vorstandssitzung \qs{\name\ e.V.}}
{\LARGE Protokoll - Vorstandssitzung \qs{\name\ e.V.}}

\section*{Dokumentenstatus}
\documentstatus\\
Fassung vom \eventdate\ (\revision)

\section*{1) {\"U}berblick}
    Ort: Sublab e.V., Karl-Heine-Str. 93, Leipzig\\
    Datum: \eventdate\\
    Anzahl der anwesenden Vorstandsmitglieder: 2\\
    Anzahl der anwesenden Beisitzer: 3\\
    Schriftf{\"u}hrer: \schriftfuehrer
\subsection*{Beschlussf{\"a}higkeit}
    Es sind 7 von 9 Stimmgewichte anwesend. Der Vorstand ist beschlussf{\"a}hig.

\section*{2) Tagesordnung}
    Der folgende Punkt steht auf der Tagesordnung der Sitzung:
	\begin{itemize}
        \item Beschluss zur Unterschrift des Mietvertrages für die Räume in der \qs{Dreilindenstr. 19}
    \end{itemize}

\section*{3) Sichtung des Mietvertrages}
	Der Mietvertrag lag den Vorstandsmitgliedern seit 2017-07-22 vor. Einzelne Punkte
	des Mietvertrages wurden w{\"a}hrend der Vorstandssitzung diskutiert.

\section*{4) Abstimmung {\"u}ber die Unterschrift des Mietvertrages}
    \begin{itemize}
        \item Der Mietvertrag wird von den Vorstandsmitgliedern akzeptiert und die Unterschrift einstimmig beschlossen mit 7 Stimmgewichten daf{\"u}r, 0 dategen, 0 enthalten.\\
    \end{itemize}

\section*{5) Ende der Sitzung}
    Mit der Unterschrift best{\"a}tigt der Schriftf{\"u}hrer die Inhalte dieses Protokolls.
\\
\\
\\
\\
Leipzig, den \eventdate \ \ \ Schriftf{\"u}hrer: \schriftfuehrer
\end{document}
