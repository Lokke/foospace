\documentclass[10pt,a4paper]{scrartcl}
\usepackage{eurosym}
\usepackage[margin=3cm]{geometry}
\usepackage[utf8]{inputenc}
\newcommand{\qt}[1]{\glq\emph{#1}\grq}
\newcommand{\qs}[1]{"#1"}
\newcommand{\name}{dezentrale}
\newcommand{\revision}{$Revision: 2018-01-14$}
\newcommand{\eventdate}{14.01.2018}
\newcommand{\schriftfuehrer}{Jan Hollburg}
\newcommand{\documentstatus}{Ver{\"o}ffentlicht (PUBLIC)}
\setlength{\parskip}{6pt}
\setlength{\parindent}{0pt}
\usepackage{enumerate}
\usepackage{color}
\pagestyle{plain}
\usepackage{palatino}
\usepackage[bookmarks,bookmarksopen=true,bookmarksnumbered=true,colorlinks,linkcolor=black,urlcolor=blue]{hyperref}
\begin{document}
\title{Protokoll - Vorstandssitzung \qs{\name\ e.V.}}
{\LARGE Protokoll - Vorstandssitzung \qs{\name\ e.V.}}

\section*{Dokumentenstatus}
\documentstatus\\
Fassung vom \eventdate\ (\revision)

\section*{1) {\"U}berblick}
    Ort: dezentrale e.V., Dreilindenstr. 19, 04177 Leipzig\\
    Datum: \eventdate\\
    Anzahl der anwesenden Vorstandsmitglieder: 3\\
    Anzahl der anwesenden Beisitzer: 1\\
    Schriftf{\"u}hrer: \schriftfuehrer
\subsection*{Beschlussf{\"a}higkeit}
    Es sind 7 von 9 Stimmgewichte anwesend. Der Vorstand ist beschlussf{\"a}hig.

\section*{2) Tagesordnung}
	Die folgenden Punkte stehen auf der Tagesordnung der Sitzung:
	\begin{itemize}
		\item Mitgliedsantrag-Bearbeitung
		\item Mitgliederliste als Datenbank
		\item dezentrale e.V. - bezogene Zug{\"a}nge und Zugangsdaten
		\item Stromvertrag mit der Holzwerkstatt
		\item J{\"a}hrliche Mitgliederversammlung
		\item Finanz-Report
		\item Spendenquittungen, Mitgliedsbeitragsquittungen

	\end{itemize}

\section*{3) Mitgliedsantrag-Bearbeitung}
	Es wurde folgendes Verfahren zur Bearbeitung von Mitgliedsantr{\"a}gen ausgearbeitet:\\
	Vorbedingung: Der Antragsteller muss sich min. einem Vorstandsmitglied pers{\"o}nlich vorgestellt haben.\\
	\begin{itemize}
		\item Mitgliedsantrag unterschrieben entgegennehmen
		\item Der Antragsteller kann ab diesem Zeitpunkt einen inaktiven Zugangstoken (gegen Pfand) erhalten. Die Tokens werden dazu in der dezentrale vorgehalten.
		\item Der Schriftf{\"u}hrer erfasst den Antrag elektronisch als Scan und legt einen Eintrag in der Mitgliedsdatenbank an. Dabei wird die Mitgliedsnummer generiert.
		\item Der Schriftf{\"u}hrer versendet eine Antwort an das neue Mitglied welche Mitgliedsnummer, Kontodaten, ggf. Token-ID usw. enth{\"a}lt.\\
		Es wird ein LateX-Template erstellt, um eine PDF für die Antwort zu generieren.
		\item Der Schriftf{\"u}hrer informiert den Vorstand {\"u}ber das neue Mitglied durch Mail an vorstand@dezentrale.space
		\item Ab jetzt ist der Antragsteller \qs{vorl{\"a}ufiges Mitglied}
		\item Der Schatzmeister {\"u}berpr{\"u}ft den Zahlungseingang (Mitgliedsbeitrag) und aktualisiert den Datenbankeintrag
		\item Der Zugangstoken wird dadurch aktiviert.
		\item F{\"u}r die {\"U}berführung der \qs{vorl{\"a}ufigen} in \qs{vollwertige} Mitgliedschaft ist es n{\"o}tig:\\
		\begin{itemize}
			\item Dass der Mitgliedsbeitrag bezahlt wurde
			\item Dass eine 7-Tage-Frist ohne Veto durch den Vorstand vergangen ist
		\end{itemize}
	\end{itemize}

\section*{4) Mitgliederliste als Datenbank}
Für die Verwaltung der Mitgliedsdaten soll JVerein evaluiert werden.\\
Weiterhin wurden Felder f{\"u}r einen m{\"o}glichen Datenbankentwurf definiert.


\section*{5) dezentrale e.V. - bezogene Zug{\"a}nge und Zugangsdaten}
Es wurde begonnen, alle offiziellen Accounts des Vereins zu sammeln und die Zugangsdaten beim Vorstand zu hinterlegen.

\section*{6) Stromvertrag mit Holzwerkstatt}
	Es wird kein separater Vertrag erstellt, jedoch eine Rechnung für 2017 nach Z{\"a}hlerstand erstellt.
	Die Z{\"a}hlergrundgeb{\"u}hr wird zu 50% berechnet.\\
	Ab 2018 wird ein Abschlag abgesch{\"a}tzt, der monatlich per {\"U}berweisung zu bezahlen ist.
	Der Stromverbrauchsz{\"a}hler ist der Holzwerkstatt und der dezentrale zug{\"a}ngig.
	Auf Wunsch kann der tats{\"a}chliche Stromverbrauch ermittelt und der Abschlag angepasst werden.
	Sp{\"a}testens zum Jahreswechsel wird der tats{\"a}chliche Stromverbrauch ermittelt und Differenzbetr{\"a}ge ausgeglichen.
	Die Grundgeb{\"u}hr des Stromanbieters wird weiterhin zu gleichen Teilen aufgeteilt.

\section*{7) J{\"a}hrliche Mitgliederversammlung}
	Der Termin wird auf den 04.03.2018 festgesetzt. Es wird eine vorl{\"a}ufige Einladung geben sowie eine formale,
	fristgerechte, sobald die Satzungs{\"a}nderungen ausgearbeitet sind.\\
	Themen für die MV 2018-03:\\
	\begin{itemize}
	\item Bericht des Schatzmeisters
	\item Mitgliederanzahl, Mitgliederentwicklung
	\item Aufz{\"a}hlung laufender Projekte
	\item m{\"o}gliche, geplante Veranstaltungen
	\item Projekte Finanzierung
	\item M{\"o}gliche Satzungs{\"a}nderungen\\
		\begin{itemize}
		\item Anpassungen hinsichtlich Gemeinn{\"u}tzigkeit (Kurz und knapp)
		\item Definition des Hausrechtes f{\"u}r alle Mitglieder
		\item Anzuwendendes Wahlverfahren {\"u}berpr{\"u}fen und festschreiben f{\"u}r anfallende Abstimmungen
		\item \qs{vorl{\"a}ufiges Mitglied}
		\item \S8 Punkt 7.: \qs{Eine Wiederwahl ist zul{\"a}ssig.} Kl{\"a}ren, ob das Wort \qs{Eine} problematisch ist.
		\end{itemize}
	\end{itemize}
\section*{8) Finanz-Report}
	Der Schatzmeister hat eine {\"U}bersicht über den Kontostand und ausstehende Mitgliedsbeitr{\"a}ge gegeben.

\section*{9) Spendenquittungen, Mitgliedsbeitragsquittungen}
	Der Schatzmeister plant f{\"u}r jedes Mitglied eine Mitgliedsbeitragsquittung des letzten Jahres vorzubereiten.

\section*{10) Ende der Sitzung}
    Mit der Unterschrift best{\"a}tigt der Schriftf{\"u}hrer die Inhalte dieses Protokolls.
\\
\\
\\
\\
Leipzig, den \eventdate \ \ \ Schriftf{\"u}hrer: \schriftfuehrer
\end{document}
