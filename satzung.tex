\documentclass[10pt,a4paper]{scrartcl}
\usepackage{eurosym}
\usepackage[utf8x]{inputenc}
\usepackage{ngerman}
\usepackage[ngerman]{babel}
\usepackage[margin=3cm]{geometry}
\newcommand{\qt}[1]{\glq\emph{#1}\grq}
\newcommand{\qs}[1]{\glqq#1\grqq}
\newcommand{\name}{FOOSPACE}
\newcommand{\revision}{$Revision: 1 $}
\setlength{\parskip}{6pt}
\setlength{\parindent}{0pt}
\usepackage{enumerate}
\usepackage{color}
\pagestyle{plain}
\usepackage{palatino}
\usepackage[bookmarks,bookmarksopen=true,bookmarksnumbered=true,colorlinks,linkcolor=black,urlcolor=blue]{hyperref}
\begin{document}

{\LARGE Satzung des \name\ e.V.}

Fassung vom \today (\revision)

\subsection*{Präambel}

Der Einzug von digitaler Informationsverarbeitung ins Alltagsleben
unserer Gesellschaft hat eine Vielzahl von Möglichkeiten geschaffen.
Der Umgang von Mensch mit Maschine und die schon als Eigenleben
qualifizierbare Beweglichkeit von Daten bergen viele Chancen,
erfordern aber auch spezielle Fähigkeiten und ein Bewusstsein für
Risiken und Gefahren.

Wir sind eine Gemeinschaft von Lebewesen, unabhängig von Alter,
Geschlecht, Herkunft, politischer Ausrichtung und gesellschaftlicher
Stellung, die im Umgang mit Informationstechnologie bildet, damit
zusammenhängende Formen von Kunst und Kultur fördert und sich für
eine jedem Wesen gerechte Entwicklung der Informationsgesellschaft
einsetzt.

\subsection*{§ 1 Name, Sitz, Geschäftsjahr}
\begin{enumerate}
\item Der Verein trägt den Namen \qs{\name}. Der Verein wird in das
Vereinsregister beim Amtsgericht Leipzig eingetragen
und der Name dann um den Zusatz \qs{e.V.} ergänzt.
\item Der Verein hat seinen Sitz in Leipzig. Das Geschäftsjahr entspricht
dem Kalenderjahr.
\end{enumerate}

\subsection*{§ 2 Zweck, Gemeinnützigkeit}
\begin{enumerate}
\item Der Verein verfolgt ausschließlich und unmittelbar gemeinnützige
Zwecke im Sinne des Abschnitts \qs{steuerbegünstigte Zwecke} der
Abgabenordnung 1977 (§51 ff AO) in der jeweils gültigen Fassung;
er dient ausschließlich und unmittelbar den in Absatz 2 angegebenen
Zwecken. Er darf keine Gewinne erzielen, er ist selbstlos
tätig und verfolgt nicht in erster Linie eigenwirtschaftliche Zwecke.
Die Mittel des Vereins werden ausschließlich und unmittelbar zu den
satzungsgemäßen Zwecken verwendet. Die Mitglieder erhalten keine
Zuwendung aus den Mitteln des Vereins. Die Mitglieder dürfen bei
ihrem Ausscheiden oder bei Auflösung des Vereins keine Anteile des
Vereinsvermögens erhalten. Niemand darf durch Ausgaben, die dem
Zwecke des Vereins fremd sind, oder durch unverhältnismäßig hohe
Vergütungen begünstigt werden.

\item Insbesondere im, jedoch nicht begrenzt auf den Rahmen der folgenden Mittel:
	\begin{itemize}
		\item Aufbau einer Begegnungsstätte fur Veranstaltungen, Experimente, kommunikativem Austausch, etc.
		\item Regelmäßige öffentliche Treffen und Informationsveranstaltungen
		\item Veranstaltungen und/oder Förderung internationaler Kongresse, Treffen
		\item Öffentlichkeitsarbeit und Telepublishing in allen Medien
		\item Förderung des schöpferisch-kritischen Umgangs mit Technologie
		\item Bildung und Weiterbildung zu technischen Fragen
	\end{itemize}
	setzt sich der Verein ein für die Förderung von:
	\begin{itemize}
		\item Erziehung, Volksbildung und Studentenhilfe (AO §52 2.7) in den in der Präambel angesprochenen Themen,
			insbesondere der digitalen Informationsverarbeitung und deren Einfluss auf die Gesellschaft;
			durchgefuhrt durch Bildungsveranstaltungen, Experimentierräume und -projekte sowie Informationsaustausch.
		\item Kunst und Kultur (AO §52 2.5.) in bestehenden und neuen Formen, wie sie durch Einflusse der digitalen
			Informationsverarbeitung entstanden sind und entstehen, z.\, B. NetArt, BlinkenLights und andere Computerkunst.
		\item Internationaler Gesinnung und Volkerverständigung (AO §52 2.13.) durch Austausch mit ähnlichen oder gleichgesinnten
			Vereinen, Einrichtungen und Projekten.
		\item Kriminalprävention (AO §52 2.20.) insbesondere im Umgang mit digitaler Informationsverarbeitungstechnik durch
			Aufklärung über rechtliche Grundsätze, angemessene Verhaltensweisen und Unterbreitung von Alternativen zu 
			kriminellen Handlungsweisen
		\item Demokratischem Staatswesen (AO §52 2.24.) im Besonderen im Zusammenhang mit der Entwicklung der Gesellschaft 
			zu einer Informationsgesellschaft durch Veranstaltungen und Diskussionen zu Themen wie Urheberrecht, Datenschutz, 
			Netzneutralität, freie und offene Software, etc.
	\end{itemize}
\end{enumerate}
\subsection*{§ 3 Mitgliedschaft}
\begin{enumerate}
	\item Mitglied des Vereins kann jede natürliche oder juristische Person oder
		nicht rechts\-fähiger Verein werden, die seine Ziele unterstützt.
	\item Über die Aufnahme von Mitgliedern entscheidet der Vorstand.
	\item Die Beitrittserklärung erfolgt schriftlich gemäß §11 gegenüber dem Vorstand. Die
		Mitgliedschaft beginnt mit der Aushändigung einer entsprechenden Bestätigung durch
		ein Vorstandsmitglied.
	\item Hat der Vorstand die Aufnahme abgelehnt, so kann der Mitgliedschaftsbewerber Einspruch
		zur nächsten Mitgliederversammlung einlegen, die dann abschließend über die Aufnahme
		oder Nichtaufnahme entscheidet.
	\item Die Mitgliedschaft endet durch Austrittserklärung, durch Ausschluss, durch Tod von
		natür\-li\-chen Personen oder durch Auflösung und Erlöschung von nicht
		natür\-lichen Personen.
		Die Beitragspflicht für das laufende Geschäftsjahr wird von der Geschäftsordnung
		geregelt.
	\item Der Austritt wird durch gemäß §11 schriftliche Willenserklärung gegenüber dem Vorstand
		erklärt.
\end{enumerate}
\subsection*{§ 4 Ausschluss eines Mitglieds }
\begin{enumerate}
	\item Ein Mitglied kann durch Beschluss des Vorstandes ausgeschlossen werden, wenn es das
		Ansehen des Vereins schädigt, sich innerhalb des Vereines wiederholt respektlos oder beleidigend äussert, sich gruppenbezogen Menschenfeindlich betätigt, seinen Beitragsverpflichtungen nicht nachkommt oder
		wenn ein sonstiger wichtiger Grund vorliegt. Der Vorstand muss dem auszuschließenden
		Mitglied den Beschluss in schriftlicher Form gemäß §11 unter Angabe von Gründen
		mitteilen und ihm auf Verlangen eine Anhörung gewähren.
	\item Gegen den Beschluss des Vorstandes ist die Anrufung der Mitgliederversammlung
		zu\-läs\-sig. Bis zum Beschluss der Mitgliederversammlung ruht die Mitgliedschaft.
	
	%[FIXME] Ausschluss ohne Gründe durch die (grosse) MV
\end{enumerate}
\subsection*{§ 5 Rechte und Pflichten der Mitglieder}
\begin{enumerate}
	\item Die Mitglieder sind berechtigt, die Leistungen des Vereins entsprechend der vorhandenen
		Möglichkeiten und in angemessenem und verhältnismäßigem Ausmaß in Anspruch zu nehmen.
	\item Die Mitglieder sind verpflichtet, die satzungsgemäßen Zwecke des Vereins zu unterstützen
		und zu fördern.
	\item Der Verein erhebt einen Mitgliedsbeitrag, zu dessen Zahlung die Mitglieder verpflichtet
		sind. Näheres regelt eine Geschäftsordnung, die von der Mitgliederversammlung beschlossen
		wird.
	% [FIXME] Pflicht die Räume und Einrichtungen pfleglich und ordentlich zu behandeln.
\end{enumerate}
\subsection*{§ 6 Organe des Vereins }
\begin{enumerate}
	\item Die Organe des Vereins sind:
		\begin{itemize}
			\item Die Mitgliederversammlung
			\item Der Vorstand
		\end{itemize}
\end{enumerate}
\subsection*{§ 7 Mitgliederversammlung}
\begin{enumerate}
	\item Die Mitgliederversammlung ist das oberste Beschlussorgan des Vereins. Ihr obliegen
		alle Entscheidungen, die nicht durch die Satzung oder die Geschäftsordnung einem
		anderen Organ übertragen wurden.
	\item Beschlüsse werden von der Mitgliederversammlung durch öffentliche Abstimmung getroffen.
		Auf Wunsch eines Mitglieds ist geheim abzustimmen.
	\item Jedes Mitglied hat genau eine Stimme.
	\item Zur Fassung eines Beschlusses ist eine einfache Mehrheit der abgegebenen Stimmen
		notwendig. Ausgenommen sind die in §9 und §10 geregelten Angelegenheiten. Eine
		zur Herstellung der Beschlussfähigkeit nötige Untergrenze von abgegebenen Stimmen
		wird in der Geschäftsordnung festgelegt.
	\item Eine ordentliche Mitgliederversammlung, bezeichnet als Jahreshauptversammlung,
		wird einmal jährlich einberufen. Ihre Tagesordnung umfasst unter anderem die
		Vorstellung des Rechenschaftsberichts für das vorherige Geschäftsjahr durch
		den Schatzmeister.
	\item Eine außerordentliche Mitgliederversammlung kann jederzeit einberufen werden, wenn
		mindestens 23\% der Mitglieder oder der Vorstand dies jeweils schriftlich gemäß §11
		unter Angabe eines Grunds beantragen. Dem angegebenen Grund müssen die gewünschten
		Tagesordnungspunkte zu entnehmen sein; sie werden auf die Einladung übernommen.
	\item Dem Vorstand obliegt zu allen Mitgliederversammlungen die Festsetzung eines Termins
		und die rechtzeitige Einladung aller Mitglieder bis spätestens 2 Wochen vor dem
		von ihm festgesetzten Termin. Bei von den Mitgliedern beantragten
		Mitgliederversammlungen darf der Termin nicht mehr als 8 Wochen nach dem Eingang
		des Antrags beim Vorstand liegen.
	\item Der Vorstand kann die Einladungen auf schriftlichem Weg gemäß §11 zustellen, muss
		jedoch eine Kopie auf dem Postweg zustellen, falls das Mitglied den Wunsch dazu
		schriftlich gemäß §11 angemeldet hat.
	\item In der Einladung werden die Tagesordnungspunkte sowie weitere nötige Informationen
		bekannt gegeben. Die Mitgliederversammlung kann per Beschluss die Tagesordnung
		ver\-ändern.
	\item Über die Beschlüsse der Mitgliederversammlung ist ein Protokoll anzufertigen,
		das vom Versammlungsleiter und dem Schriftführer zu unterzeichnen ist.
		Das Protokoll ist innerhalb von 14 Tagen allen Mitgliedern zugänglich zu
		machen und auf der nächsten Mitgliederversammlung genehmigen zu lassen.
	\item Der Vorstandsvorsitzende ist Versammlungsleiter der Mitgliederversammlung.
		Die Mitgliederversammlung kann durch Beschluss einen anderen Versammlungsleiter
		oder Schrift\-füh\-rer bestimmen.
\end{enumerate}
\subsection*{§ 8 Vorstand }
\begin{enumerate}
    \item Der Vorstand besteht aus mindestens drei Mitgliedern: dem
    Vorstandsvorsitzenden, dem Schatzmeister und dem Schriftführer. Des
    Weiteren ein Vertreter des Vorstandsvorsitzenden und bis zu drei Beisitzer in den Vorstand gewählt werden. Es
    kann auf Wunsch der Mitglieder auf eine Wahl des Vertreters des Vorstandsvorsitzenden und der Beisitzer verzichtet
    werden.
    \item Vorstand im Sinne des § 26 BGB ist jedes Vorstandsmitglied.
    Vorstandsvorsitzender, Vertreter des Vorstandsvorsitzenden, Schatzmeister und Schriftführer sind einzeln
    berechtigt, den Verein nach außen zu vertreten. Die Geschäftsordnung kann
    hierfür Einschränkungen festlegen.
	\item Der Schatzmeister über\-wacht die Haushaltsführung und verwaltet das
		Ver\-mö\-gen des Vereins. Nä\-her\-es regelt die Ge\-schäfts\-ord\-nung.
	\item Vorstandsmitglieder können jederzeit von ihrem Amt zurücktreten.
	\item Bei Rücktritt oder andauernder Ausübungsunfähigkeit eines Vorstandsmitglieds ist
		der gesamte Vorstand neu zu wählen. Bis zur Wahl eines neuen Vorstands ist der
		bisherige Vorstand zur bestmöglichen Wahrnehmung seiner Aufgaben verpflichtet.
    \item Die Amtsdauer der Vorstandsmitglieder beträgt zwei Jahre. Sie werden
    von der Mitgliederversammlung aus den Mitgliedern des Vereins gewählt. Es
    werden nacheinander Vorstandsvorsitzender, Schatzmeister und Schriftführer
    sowie falls gewünscht Vertreter des Vorstandsvorsitzenden und bis zu drei Beisitzer gewählt. Eine Wiederwahl ist
    zulässig.
	\item Der Vorstand ist Dienstvorgesetzter aller vom Verein angestellten Mitarbeiter;
		er kann diese Aufgabe einem Vorstandsmitglied übertragen.
	\item Die Vorstandsmitglieder sind grundsätzlich ehrenamtlich tätig; sie haben Anspruch
		auf Erstattung notwendiger Auslagen, deren Rahmen von der Geschäftsordnung
		festgelegt wird.
    \item Der Vorstand tritt nach Bedarf zusammen. Die Vorstandssitzungen
    werden vom Schriftführer schriftlich einberufen. Der Vorstand ist
    beschlussfähig wenn mindestens zwei Drittel der Vorstandsmitglieder
    anwesend sind. Die Beschlüsse der Vorstandssitzung sind schriftlich zu
    protokollieren. Protokolle sind auf Anfrage den Vereinsmitgliedern zugänglich zu machen.
    \item Vorstandsvorsitzender, Vertreter des Vorstandsvorsitzenden, Schatzmeister und Schriftführer haben bei
    Abstimmungen des Vorstands jeweils zwei Stimmen. Jeder Beisitzer hat eine
    Stimme. Bei Abstimmungen ist eine Mehrheit von zwei Dritteln der
    abgegebenen gültigen Stimmen nötig.
\end{enumerate}
\subsection*{§ 9 Satzungs- und Geschäftsordnungsänderung}
\begin{enumerate}
	\item Über Satzungs- und Geschäftsordnungsänderungen kann in der Mitgliederversammlung
		nur abgestimmt werden, wenn auf diesen Tagesordnungspunkt hingewiesen wurde und der
		Einladung sowohl der bisherige als auch der vorgesehene neue Text beigefügt
		worden war.
	\item Für die Satzungs- oder Geschäftsordnungsänderung ist eine Mehrheit von zwei
		Dritteln in der Mitgliederversammlung erforderlich.
	\item Satzungsänderungen, die von Aufsichts-, Gerichts- oder Finanz\-be\-hör\-den aus formalen
		Grün\-den verlangt werden, kann der Vorstand von sich aus vornehmen. Diese
		Sat\-zungs\-än\-der\-ung\-en müs\-sen der nächs\-ten Mitgliederversammlung mitgeteilt
		werden.
\end{enumerate}
\subsection*{§ 10 Auflösung des Vereins und Vermögensbindung}
\begin{enumerate}
	\item Die Auflösung des Vereins muss von der Mitgliederversammlung mit einer Mehrheit von
		zwei Dritteln beschlossen werden. Die Abstimmung ist nur möglich, wenn auf der Einladung
		zur Mitgliederversammlung als einziger Tagesordnungspunkt die Auflösung des Vereins
		angekündigt wurde.
	\item Bei Auflösung des Vereins, Aufhebung der Körperschaft oder Wegfall der
		gemeinnützigen Zwecke darf das Vermögen der Körperschaft nur für
		steuerbegünstigte Zwecke verwendet werden. Zur Erfüllung dieser
		Voraussetzung wird das Vermögen einer anderen steu\-er\-be\-güns\-tig\-ten
		Körperschaft oder einer Körperschaft öffentlichen Rechts für
		steuerbegünstigte Zwecke über\-tra\-gen, die ebenfalls den Auftrag
		zur Bildung und Volksbildung im Umgang mit Informationstechnologie
		wahrnimmt. Nä\-he\-res kann die Geschäftsordnung regeln.
	\item Der Grundsatz der Vermögensbindung ist bei der Fassung von
		Beschlüssen über die künf\-ti\-ge Verwendung des Vereinsvermögens zwingend
		zu erfüllen.
	\item Bei Verlust der Anerkennung als gemeinnütziger Verein gelten die vorgenannten Absätze analog, 
		das Vermögen und die Güter des Vereins werden entsprechend übertragen.
\end{enumerate}
\subsection*{§ 11 Schriftform, Abstimmungsfähigkeit}
\begin{enumerate}
	\item Schriftliche Erklärungen im Sinne dieser Satzung können auch
		elektronische Dokumente sein. Die Geschäftsordnung bestimmt
		Anforderungen, Zustellwege und Zuordnung derartiger Dokumente.
	\item Zu Mitgliederversammlungen werden elektronisch nach Abs. 1 oder
		postalisch zugestellte Stimmen von Mitgliedern wie Stimmen
		von anwesenden Mitgliedern gezählt.
\end{enumerate}

\end{document}
