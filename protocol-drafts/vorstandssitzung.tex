\documentclass[10pt,a4paper]{scrartcl}
\usepackage{eurosym}
\usepackage[utf8x]{inputenc}
\usepackage{ngerman}
\usepackage[ngerman]{babel}
\usepackage[margin=3cm]{geometry}
\newcommand{\qt}[1]{\glq\emph{#1}\grq}
\newcommand{\qs}[1]{\glqq#1\grqq}
\newcommand{\name}{dezentrale}
\newcommand{\revision}{$Revision: 2017-06-21$}
\newcommand{\eventdate}{21.06.2017}
\newcommand{\sitzungsleiter}{(regul{\"a}rer Vorstandsvorsitzender des Vereins, oder i.V. Name)}
\newcommand{\schriftfuehrer}{(regul{\"a}rer Schriftf{\"u}hrer des Vereins, oder i.V. Name)}
\newcommand{\documentstatus}{Entwurf (DRAFT)}
\setlength{\parskip}{6pt}
\setlength{\parindent}{0pt}
\usepackage{enumerate}
\usepackage{color}
\pagestyle{plain}
\usepackage{palatino}
\usepackage[bookmarks,bookmarksopen=true,bookmarksnumbered=true,colorlinks,linkcolor=black,urlcolor=blue]{hyperref}
\begin{document}
\title{Protokoll - Vorstandssitzung \qs{\name\ e.V.}}
{\LARGE Protokoll - Vorstandssitzung \qs{\name\ e.V.}}

\subsection*{Dokumentenstatus}
\documentstatus\\
Fassung vom \eventdate\ (\revision)

\subsection*{1) {\"U}berblick}
    Ort: Vereinsr{\"a}umlichkeiten des \name\ e.V.\\
    Datum: \eventdate\\
	Anzahl der anwesenden Vorstandsmitglieder: \\
	Anzahl der anwesenden Beisitzer: \\
    Sitzungsleiter: \sitzungsleiter\\
    Schriftf{\"u}hrer: \schriftfuehrer

\subsection*{2) Tagesordnung}
    Die folgenden Punkte stehen auf der Tagesordnung der Sitzung:
	\begin{itemize}
        \item Foo
        \item Bar
        \item Baz
    \end{itemize}

\subsection*{3) Foo}
    Foo-Content

\subsection*{4) Bar}
    Bar-Content

\subsection*{5) Baz}
    Baz-Content

\subsection*{6) Ende der Sitzung}
    Mit ihren Unterschriften best{\"a}tigen der Sitzungsleiter und der Schriftf{\"u}hrer die Inhalte dieses Protokolls.
\\
\\
\\
\\
\\
Leipzig, den \eventdate \ \ \ Sitzungsleiter: \sitzungsleiter\\
\\
\\
\\
\\
\\
\\
Leipzig, den \eventdate \ \ \ Schriftf{\"u}hrer: \schriftfuehrer
\end{document}
