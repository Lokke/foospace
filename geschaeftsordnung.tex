\documentclass[10pt,a4paper]{article}
\usepackage{ucs}
\usepackage{eurosym}
\usepackage[utf8x]{inputenc}
\usepackage[ngerman]{babel}
\usepackage[margin=3cm]{geometry}
\newcommand{\qt}[1]{\glq\emph{#1}\grq}
\newcommand{\qs}[1]{\glqq#1\grqq}
\newcommand{\name}{FOOSPACE}
\newcommand{\revision}{$Revision: 1$}
\setlength{\parskip}{6pt}
\setlength{\parindent}{0pt}
\usepackage{enumerate}
\usepackage{color}
\pagestyle{plain}
\usepackage{palatino}
\usepackage[bookmarks,bookmarksopen=true,bookmarksnumbered=true,colorlinks,linkcolor=black,urlcolor=blue]{hyperref}
\usepackage{ulem}
\begin{document}

{\LARGE Geschäftsordnung des \name\ e.V.}

Fassung vom \today (\revision)

\subsection*{§ 1 Mitgliedsbeträge}
% 5.3
\begin{enumerate}
\item Der Verein erhebt gemäß §5 seiner Satzung Mitgliedsbeiträge wie folgt:
	\begin{itemize}
	\item 12,00€/Monat für ermäßigte Mitgliedschaft
	\item 23,00€/Monat für normale Mitgliedschaft
	\item 42,00€/Monat, 64,00€/Monat, 128,00€/Monat oder 256,00€/Monat für Nerdmitglieder
	\end{itemize}
\item Voraussetzung für die ermäßigte Mitgliedschaft ist die Vorlage eines
	Schüler-, Studenten- oder Rentenausweises, eines ALGII-Bescheides,
	eines Sozialhilfebescheids, eines Bescheids des Arbeitsamtes
	oder eines vergleichbaren Nachweises gegenüber dem Vorstand.
	Der entsprechende Nachweis ist jährlich neu zu erbringen.
\item Die Nerdmitgliedschaft steht jedem Mitglied zur Wahl, das den Verein stärker
	finanziell unterstützen möchte. Mit einer solchen
	Nerdmitgliedschaft sind keinerlei Privilegien oder Stimmvorteile gegenüber
	den anderen beiden Mitgliedschaften verbunden.
\item Änderungen bezüglich der Mitgliedschaft (Mitgliedbeitrag oder Mitgliedsart)
	sind dem Vorstand schriftlich mitzuteilen. Sofern nicht anders
	vereinbart gelten die gleichen Fristen wie in dieser Ordnung unter §2
	zum Austritt angegeben.
\end{enumerate}

\subsection*{§ 2 Verpflichtungen nach Austritt}
% 3.5
\begin{enumerate}
\item Das Ende der Mitgliedschaft eines Mitglied entbindet dieses nicht von der
	Beitragsverpflichtung bis zum nächsten Quartalsende.
\end{enumerate}

\subsection*{§ 3 Einschränkungen der Verfügungsberechtigung des Vorstands}
% 8.2
\begin{enumerate}
\item Vorstandsmitglieder, die den Verein alleine nach außen vertreten
    dürfen, sind bei Rechtsgeschäften bis zu einem Betrag von 500 EUR
    verfügungsberechtigt. Über einen Betrag von bis zu 5000 EUR muss der
    Vorstand abstimmen. Bei höheren Beträgen ist ein Beschluss durch die
    Mitgliederversammlung nötig.
\end{enumerate}

\subsection*{§ 4 Beschlussfähigkeit der Mitgliederversammlung}
% 7.4.
\begin{enumerate}
\item Die Untergrenze für die Beschlussfähigkeit der Mitgliederversammlung
	gemäß Satzung §7 beträgt 23\% der Mitglieder.
\end{enumerate}

\subsection*{§ 5 Grundsätze der Vermögensverwaltung des Vereins}
\begin{enumerate}
\item Die Summe der Ausgaben eines Jahres darf das liquide Vereinsvermögen
	nicht übersteigen.
\end{enumerate}

\subsection*{§ 6 Aufgaben des Schatzmeisters}
\begin{enumerate}
\item Der Schatzmeister hat auf eine sparsame und wirtschaftliche
	Haushaltsführung hinzuwirken
\item Der Schatzmeister legt nach Eintragung des Vereines in das Vereinsregister
	ein Konto auf den Namen des Vereines an und verwaltet dort das
	Vereinsvermögen.
\item Für Abhebungen vom Vereinskonto ist die Unterschrift von zwei
	Vorstandsmitgliedern nötig.
	% [FIXME] Obsolete durch Geldautomaten?
\item Der Schatzmeister informiert die Vereinsmitglieder mindestens
	vierteljährlich sowie innerhalb von sechs Wochen nach größeren
	Veranstaltungen, bei denen der Verein als Veranstalter oder
	Mitveranstalter auftritt, über den Kassenstand. Einnahmen und
	Ausgaben über 100 EUR sind dabei einzeln aufzulisten.
\item Als Vorstandsmitglied hat der Schatzmeister die Einbringung der
	Mitgliedsbeiträge und anderer Einnahmen zu organisieren. Dabei
	genießt er die volle Unterstützung des Vorstands.
\item Für laufende Einnahmen und Ausgaben führt der Schatzmeister eine
	Bargeldkasse. Über\-schüs\-sige Bargeldsummen werden von ihm
	regelmäßig auf dem Vereinskonto abgelegt.
\item Für Bareingänge stellt der Schatzmeister eine formgerechte Quittung
	in doppelter Ausfertigung aus, davon eine für den Einzahler.
\item Der Schatzmeister legt ein geeignetes Vermögensregister an, das
	nach den Regeln der einfachen Buchführung zu führen ist und aus
	folgenden Teilen besteht:
	\begin{itemize}
	\item Kassenbuch für die Bargeldkasse
	\item Hauptbuch für das Vereinskonto
	\item Inventarliste für Vermögensgegenstände 
	\end{itemize}
\item Jede einzelne Ausgabe muss belegt werden. Jeder Beleg muss von
	dem Vereinsmitglied, das die Ausgabe getätigt hat, umgehend
	beim Schatzmeister eingereicht werden.
\item Sollten Güter zugunsten des Vereins eingehen, sind diese im
	Vermögensregister einzutragen. Der Schatzmeister hat nach
	Genehmigung durch den Vorstand ein Aufbewahrungsprotokoll
	anzufertigen, ein Exemplar für den Besorger, eins zur
	Dokumentation beim Schatzmeister.
\item Der Schatzmeister führt die Liste der Vereinsmitglieder.
	Periodisch werden von ihm die sich ergebenden Veränderungen
	durch Zugänge und Abgänge den Vereinsmitgliedern mitgeteilt.
\item Für den Jahresabschluss oder bei Wechsel des Schatzmeisters ist
	durch ihn eine Bilanz zu erstellen. 
	% [FIXME] Sollte gestrichen werden.
	% Aus Steuergründen: *wenn* bilanziert wird, dann gibt man dadurch faktisch
	% den Anspruch auf 'einfache Buchführung' auf.
	% 'Bilanz' hat hier eine klar definierte Bedeutung. Wenn es nur um eine 'Übersicht' geht
	% reicht wahrscheinlich auch 'EÜR'.
\end{enumerate}

\subsection*{§ 7 Erstattung der Auslagen des Vorstands}
% 8.8
\begin{enumerate}
\item Auslagen des Vorstandes zur Verfolgung der Vereinszwecke werden
	in voller Höhe erstattet. Auf Beschluss der Mitgliederversammlung
	muss der Vorstand in einer Stellungnahme Zweck- und Verhältnis\-mäßigkeit
	der Ausgaben nachweisen.
\end{enumerate}

\subsection*{§ 8 Elektronische Schriftform}
% 11
\begin{enumerate}
\item Elektronische Dokumente im Sinne von §11 der Satzung sind mit PGP/GPG
	oder mit S/MIME signierte E-Mails. Jedes Mitglied kann beim Vorstand
	einen öffentlichen Schlüssel bzw. sein Zertifikat hinterlegen, dessen
	Signatur die jeweiligen E-Mails tragen müssen. Das Mitglied hat bei
	Kompromittierung des Schlüssels für Benachrichtigung des Vorstands
	zu sorgen.
\item Im Abstimmungsprozess einer Mitgliederversammlung
	übernimmt und protokolliert der Schriftführer die Überprüfung und
	Zählung der signierten E-Mails. Ferner prüft er die Anwesenheit und
	erklärt eventuell vorliegende schriftlich abgegebene Stimmen anwesender
	Mitglieder öffentlich für ungültig.
\end{enumerate}

\subsection*{§ 9 Sicherheitsbeauftragter}
\begin{enumerate}
\item Der Vorstand ernennt einen Sicherheitsbeauftragten. Seine Aufgaben
	umfassen insbesondere die Aufklärung und Information der Mitglieder
	zu Sicherheits- und Schutzmaßnahmen, gesetzlichen Regelungen und
	notwendigen Verhaltensweisen zur Vermeidung von Unfällen. Weiterhin
	überprüft er die Einhaltung dieser Regelungen in den Räumen des
	Vereins.
\end{enumerate}

\end{document}
