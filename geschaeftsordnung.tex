% [1]
%----todo:
% technische lösung um mitglieder kurzfristig zu Input zu zwingen, vor der nächsten mitgliederversammlung

\documentclass[10pt,a4paper]{article}
\usepackage{ucs}
\usepackage{eurosym}
\usepackage[utf8x]{inputenc}
\usepackage[ngerman]{babel}
\usepackage[margin=3cm]{geometry}
\newcommand{\qt}[1]{\glq\emph{#1}\grq}
\newcommand{\qs}[1]{\glqq#1\grqq}
\newcommand{\name}{dezentrale}
\newcommand{\revision}{$Revision: 2017-06-11$}
\setlength{\parskip}{6pt}
\setlength{\parindent}{0pt}
\usepackage{enumerate}
\usepackage{color}
\pagestyle{plain}
\usepackage{palatino}
\usepackage[bookmarks,bookmarksopen=true,bookmarksnumbered=true,colorlinks,linkcolor=black,urlcolor=blue]{hyperref}
\usepackage{ulem}
\begin{document}

{\LARGE Geschäftsordnung des \name\ e.V.}

Fassung vom \today (\revision)
%
% Paragraph 1 ====================================================
%
\subsection*{§ 1 Mitgliedsbeiträge und Mitgliedsarten}
% 5.3
\begin{enumerate}
\item Der Verein erhebt gemäß §5 seiner Satzung Mitgliedsbeiträge wie folgt:
	\begin{itemize}
    \item 16,00€/Monat für ermäßigte Mitgliedschaft
    \item 32,00€/Monat für normale Mitgliedschaft
	\item 42,00€/Monat, 64,00€/Monat, 128,00€/Monat oder 256,00€/Monat für Nerdmitglieder
	\end{itemize}
    

    \item Mitglieder sind verpflichtet, den Mitgliedsbeitrag bis zum dritten Werktag des Monats
    auf das Konto des Vereines zu überweisen. Die Mitgliedschaft endet bei einem Rückstand
    von zwei Monatsbeiträgen automatisch. Ausnahmen können vom Vorstand beschlossen werden.
    Bei automatischer Beendigung der Mitgliedschaft wird das Mitglied schriftlich über den
    Vorgang informiert.

    \item Vorrausetzung für die ermäßigte Mitgliedschaft erfüllen alle Personen, welche Schüler, Studenten,
    Rentner, Geflüchtete, oder Empfänger von staatlicher Hilfe sind, und dies jährlich nachweisen.

\item Die Nerdmitgliedschaft steht jedem Mitglied zur Wahl, das den Verein stärker
	finanziell unterstützen möchte. Mit einer solchen
	Nerdmitgliedschaft sind keinerlei Privilegien oder Stimmvorteile gegenüber
	den anderen beiden Mitgliedschaften verbunden.
    
    \item Ein Mitgliedschaftsanwärter hat die Pflicht, sich einem Vorstandsmitglied oder zur 
    Mitgliederversammlung persönlich vorzustellen, um die Mitgliedschaft wirksam zu machen.
    \item Unabhängig vom Mitgliedsbeitrag wird zwischen regulären Mitgliedern und Fördermitgliedern
    unterschieden. Bei Eintritt in den Verein ist jedes Mitglied ein reguläres Mitglied.
    Nimmt ein reguläres Mitglied an zwei aufeinanderfolgenden Mitgliederversammlungen nicht
    teil, wird es automatisch zum Fördermitglied. Bei Teilnahme an einer Mitgliederversammlung
    kann das Mitglied auf eigenen Wunsch wieder zur regulären Mitgliedschaft wechseln.
    Fördermitglieder haben kein Stimmrecht in der Mitgliederversammlung.

    
\item Änderungen bezüglich der Mitgliedschaft (Mitgliedbeitrag, Mitgliedsart oder Beendigung der Mitgliedschaft)
	sind dem Vorstand schriftlich mitzuteilen.
\end{enumerate}
%
% Paragraph 2 ====================================================
%
\subsection*{§ 2 Beendigung der Mitgliedschaft}
% 3.5
\begin{enumerate}
\item Eine Beendigung der Mitgliedschaft ist mit sofortiger Wirkung möglich.
\item Die Beitragszahlungspflicht endet zum entsprechenden Monatsende.
\end{enumerate}
%
% Paragraph 3 ====================================================
%
\subsection*{§ 3 Einschränkungen der Verfügungsberechtigung des Vorstands}
% 8.2
\begin{enumerate}
\item Vorstandsmitglieder, die den Verein alleine nach außen vertreten
    dürfen, sind bei Rechtsgeschäften bis zu einem Betrag von 500 EUR
    verfügungsberechtigt. Über einen Betrag von bis zu 5000 EUR muss der
    Vorstand abstimmen. Bei höheren Beträgen ist ein Beschluss durch die
    Mitgliederversammlung nötig.
\item Bei fortlaufenden Verträgen wird die erwartete Summe
    über 6 Monate analog betrachtet.
\end{enumerate}
%
% Paragraph 4 ====================================================
%
\subsection*{§ 4 Mitgliederversammlung}
% 7.4.
\begin{enumerate}
\item Vor Feststellung der Beschlussfähigkeit der Mitgliederversammlung kann ein Fördermitglied zur regulären Mitgliedschaft wechseln und damit an allen Beschlüssen der Versammlung teilnehmen.
\item Die Untergrenze für die Beschlussfähigkeit gemäß Satzung §7 beträgt 51\% der regulären Mitglieder.
\item Die teilnehmenden regulären Mitglieder sind im Versammlungsprotokoll zu protokollieren.
\item Gäste sind zur Mitgliederversammlung zugelassen.
\item Als Wahlverfahren für die Vorstandswahl wird standardmäßig das "Instant-Runoff-Verfahren" eingesetzt.
\end{enumerate}
%
% Paragraph 5 ====================================================
%
\subsection*{§ 5 Grundsätze der Vermögensverwaltung des Vereins}
\begin{enumerate}
\item Der Vorstand hat Sorge zu tragen, dass das Gesamtvermögen des Vereins nicht negativ wird.
\item Es ist anzustreben, dass die liquiden Mittel mindestens 1,5 Anteile der monatlichen durchschnittlichen Fixkosten als Reserve betragen.
\item Sollte die Mindestmenge an Liquiditätsreserve angegriffen werden, so sind darüber umgehend alle Mitglieder zu informieren.
\end{enumerate}

%
% Paragraph 6 ====================================================
%
\subsection*{§ 6 Aufgaben des Schatzmeisters}
\begin{enumerate}
\item Der Schatzmeister hat auf eine sparsame und wirtschaftliche
	Haushaltsführung hinzuwirken
\item Der Schatzmeister legt nach Eintragung des Vereines in das Vereinsregister
	ein Konto auf den Namen des Vereines an und verwaltet dort das
	Vereinsvermögen.
\item Der Schatzmeister informiert die Vereinsmitglieder mindestens
    jährlich sowie innerhalb von acht Wochen nach größeren
	Veranstaltungen, bei denen der Verein als Veranstalter oder
	Mitveranstalter auftritt, über den Kassenstand. Einnahmen und
	Ausgaben über 100 EUR sind dabei einzeln aufzulisten.
\item Als Vorstandsmitglied hat der Schatzmeister die Einbringung der
	Mitgliedsbeiträge und anderer Einnahmen zu organisieren. Dabei
	genießt er die volle Unterstützung des Vorstands.
\item Der Schatzmeister kann seine Aufgaben in eigenem Ermessen delegieren.
\item Für laufende Einnahmen und Ausgaben kann der Schatzmeister eine
    Bargeldkasse führen. Über\-schüs\-sige Bargeldsummen werden von ihm
	regelmäßig auf dem Vereinskonto abgelegt.
\item Für Bareingänge stellt der Schatzmeister eine formgerechte Quittung
	in doppelter Ausfertigung aus, davon eine für den Einzahler.
\item Der Schatzmeister legt ein geeignetes Vermögensregister an, das
	nach den Regeln der einfachen Buchführung zu führen ist und aus
	folgenden Teilen besteht:
	\begin{itemize}
	\item Kassenbuch für die Bargeldkasse
	\item Hauptbuch für das Vereinskonto
	\item Inventarliste für Vermögensgegenstände 
	\end{itemize}
\item Jede einzelne Ausgabe muss belegt werden. Jeder Beleg muss von
	dem Vereinsmitglied, das die Ausgabe getätigt hat, umgehend
	beim Schatzmeister eingereicht werden.
\item Der Schatzmeister führt die Liste der Vereinsmitglieder.
	Periodisch werden von ihm die sich ergebenden Veränderungen
	durch Zugänge und Abgänge den Vereinsmitgliedern mitgeteilt.
\item Bei Wechsel des Schatzmeisters ist durch ihn eine Übersicht zu erstellen.
\end{enumerate}
%
% Paragraph 7 ====================================================
%
\subsection*{§ 7 Erstattung der Auslagen des Vorstands}
% 8.8
\begin{enumerate}
\item Auslagen des Vorstandes zur Verfolgung der Vereinszwecke werden
	in voller Höhe erstattet. Auf Beschluss der Mitgliederversammlung
	muss der Vorstand in einer Stellungnahme Zweck- und Verhältnis\-mäßigkeit
	der Ausgaben nachweisen.
\end{enumerate}
%
% Paragraph 8 ====================================================
%
\subsection*{§ 8 Elektronische Schriftform}
% 11
\begin{enumerate}
\item Elektronische Dokumente im Sinne von §11 der Satzung sind mit PGP/GPG
	oder mit S/MIME signierte E-Mails. Jedes Mitglied kann beim Vorstand
	einen öffentlichen Schlüssel bzw. sein Zertifikat hinterlegen, dessen
	Signatur die jeweiligen E-Mails tragen müssen. Das Mitglied hat bei
	Kompromittierung des Schlüssels für Benachrichtigung des Vorstands
	zu sorgen.
\item Der Vorstand kann nach Prüfung der Sachlage auch andere Kommunikationskanäle zulassen.
\item Im Abstimmungsprozess einer Mitgliederversammlung
	übernimmt und protokolliert der Schriftführer die Überprüfung und
    Zählung der signierten elektronischen Dokumente. Ferner prüft er die physische oder fernmündliche Anwesenheit und
	erklärt eventuell vorliegende schriftlich abgegebene Stimmen anwesender
	Mitglieder öffentlich für ungültig.
\end{enumerate}
%
% Paragraph 9 ====================================================
%
\subsection*{§ 9 Infrastruktur}
% 11
\begin{enumerate}
\item Die Infrastruktur des Vereins muss hinreichend dokumentiert werden.
\end{enumerate}

\end{document}
