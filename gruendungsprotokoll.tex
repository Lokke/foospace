\documentclass[10pt,a4paper]{scrartcl}
\usepackage{eurosym}
\usepackage[utf8x]{inputenc}
\usepackage{ngerman}
\usepackage[ngerman]{babel}
\usepackage[margin=3cm]{geometry}
\newcommand{\qt}[1]{\glq\emph{#1}\grq}
\newcommand{\qs}[1]{\glqq#1\grqq}
\newcommand{\name}{dezentrale}
\setlength{\parskip}{6pt}
\setlength{\parindent}{0pt}
\usepackage{enumerate}
\usepackage{color}
\pagestyle{plain}
\usepackage{palatino}
\usepackage[bookmarks,bookmarksopen=true,bookmarksnumbered=true,colorlinks,linkcolor=black,urlcolor=blue]{hyperref}
\begin{document}
{\LARGE Protokoll - Gr{\"u}ndungsversammlung des \qs{\name\ e.V.}}
\subsection*{1. {\"U}berblick}
    Ort: Sublab e.V., Karl-Heine-Str. 93, 04229 LEIPZIG\\
    Datum: 11. Juni 2017\\
    Anzahl der Anwesenden zu Beginn: 10\\
    Versammlungsleiter: Thomas G{\"u}tt\\
    Schriftf{\"u}hrer: Jan Hollburg\\\\
\subsection*{2) Begr{\"u}{\ss}ung}
    Der Versammlungsleiter begr{\"u}{\ss}t die Anwesenden und verliest die Tagesordnung.\\\\
\subsection*{3) Vereinsname}
    Die Abstimmung zum Vereinsnamen erfolgte zuerst {\"u}ber die komplette, im Vorfeld zusammengestellte Namensliste.
	Jeder Anwesende konnte f{\"u}r jeden Namen eine Ja-Stimme geben.\\\\
\begin{table}[h!]
  \centering
  \caption{Namensabstimmung grob}
  \label{tab:table1}
  \begin{tabular}{p{6cm}||p{3cm}}
  \textbf{Name} & \textbf{Stimmen}\\
    \hline
    \hline
    Phasenraum & 2\\
    Bytezone & 0\\
    Kernel Panic & 0\\
    datensenke & 5\\
    dezentrale & 4\\
    schnittstelle & 0\\
    die Hackerei & 2\\
    Nerdpit & 2\\
    busy beaver burrow & 1\\
    Neuland & 5\\
    \qs{\ \ } & 0\\
    \hline
    \hline
  \end{tabular}
\end{table}
\\
Die Favoriten sind \qs{Phasenraum} / \qs{datensenke} / \qs{dezentrale} / \qs{Neuland}.\\
Die Entscheidung wird per Instant-Runoff-Wahl getroffen. Jede Person kann in jedem Wahlgang f{\"u}r exakt einen Namen stimmen. Der Vorschlag mit den wenigsten Stimmen wird entfernt.\\
\begin{table}[h!]
  \centering
  \caption{Instant-Runoff-Wahl des Vereinsnamens}
  \label{tab:table2}
  \begin{tabular}{p{6cm}||p{3cm}}
  \textbf{Name} & \textbf{Stimmen}\\
    \hline
    \hline
    1. Wahlgang\\
    Phasenraum & 2\\
    datensenke & 2\\
    dezentrale & 5\\
    Neuland & 0\\
    \hline
    2. Wahlgang:\\
    Phasenraum & 1\\
    datensenke & 3\\
    dezentrale & 5\\
    \hline
    3. Wahlgang:\\
    datensenke & 2\\
    dezentrale & 7\\
    \hline
    \hline
  \end{tabular}
\end{table}
\\
\\
\\
\\
    Ergebnis (siehe Tabelle 2): Es wurde Konsens erzielt, dass \qs{dezentrale} der beste Name ist.
    Der neue Verein wird demnach den Namen dezentrale, erweitert um das K�rzel \qs{e.V.} tragen.

\subsection*{4) Annahme der Satzung}
% Quick discussion regarding Selbstverstandnis:
%    we'll keep the existing Satzung's Preamble (it can affect verein's tax status, so prefer not to change this)
%    the new, more detailed Selbstverstandnis can still go on the website when finished
% https://www.phantomix.de/satzung-2017-06-11.pdf
        Die Satzung in der Fassung 2017-06-11 wird vom Versammlungsleiter verlesen, im Anschluss wird abgestimmt.
\begin{itemize}
\item Ergebnis der Abstimmung:
\item F{\"u}rstimmen: 10
\item Gegenstimmen: 0
\item Enthaltungen: 0
\end{itemize}

Die Satzung ist damit einstimmig angenommen.
\subsection*{Pause}
    Die Versammlung wird kurz pausiert
\subsection*{5) Annahme der Gesch{\"a}ftsordnung}
%Regarding infrastructure documentation
%    new paragraph has been added?
%Regarding code of conduct:
%    A) put reference to the Gesch{\"a}ftsordnung to the Code of Conduct (which is currently empty or unfinished)?
%    B) do not put any in the Gesch{\"a}ftsordnung to the Code of Conduct?
%    C) compromise suggested by equi: Gesch{\"a}ftsordnung could instead refer to ...
%    D) put this in the Hausordnung
%Regarding Vorstand replying to decisions/recommendations of the Plenum
% https://www.phantomix.de/geschaeftsordnung-2017-06-11.pdf
        Die Gesch{\"a}ftsordnung in der Fassung 2017-06-11 wird vom Versammlungsleiter verlesen, im Anschluss wird abgestimmt.
\begin{itemize}
\item Ergebnis der Abstimmung:
\item F{\"u}rstimmen: 10
\item Gegenstimmen: 0
\item Enthaltungen: 0
\end{itemize}

Die Gesch{\"a}ftsordnung ist damit einstimmig angenommen.

\subsection*{Pause}
Zu diesem Zeitpunkt hat ein Gr{\"u}ndungsmitglied die Versammlung vor�bergehend, bis zur Unterschrift der Satzung, verlassen.
\subsection*{6) Wahl des Vorstandes}
\begin{itemize}
\item Kandidaten f{\"u}r den Schatzmeister: Steven Chamberlain
\item Kandidaten f{\"u}r den Vorstandsvorsitzenden: Manuel Madrenes
\item Kandidaten f{\"u}r den Schriftf{\"u}hrer: Jan Hollburg\\\\
\end{itemize}
Die Abstimmung erfolgt nach \qs{Instant-Runoff}.\\\\
    Abstimmung zum Schatzmeister:
\begin{itemize}
\item F{\"u}r Steven Chamberlain: 8 Stimmen, 1 Enthaltung
\end{itemize}
    Abstimmung zum Vorstandsvorsitzenden:
\begin{itemize}
\item F{\"u}r Manuel Madrenes: 9 Stimmen
\end{itemize}
    Abstimmung zum Schriftf{\"u}hrer:
\begin{itemize}
\item F{\"u}r Jan Hollburg: 9 Stimmen\\\\
\end{itemize}
Der neue Vorstand akzeptiert die Wahl:
\begin{itemize}
\item Steven Chamberlain nimmt die Wahl an.
\item Manual Madrenes nimmt die Wahl an.
\item Jan Hollburg nimmt die Wahl an.\\\\
\end{itemize}
    Wahl der Beisitzer:
\begin{itemize}
\item Georg Mei{\ss}ner kandidiert als Beisitzer.
\item Johannes Schneemann kandidiert als Beisitzer.
\item Felix Pielok kandidiert als Beisitzer.
\item Thomas G{\"u}tt kandidiert als Beisitzer.\\\\
\end{itemize}
1. Beisitzerposten von 3 - 1. Wahlgang
\begin{itemize}
\item Georg Mei{\ss}ner: 5 Stimmen
\item Johannes Schneemann: 3 Stimmen
\item Felix Pielok: 0 Stimmen
\item Thomas G{\"u}tt: 0 Stimmen
\end{itemize}
    Georg Mei{\ss}ner ist damit als Beisitzer (Posten 1) gew{\"a}hlt.\\\\
2. Beisitzerposten von 3 - 1. Wahlgang
\begin{itemize}
\item Johannes Schneemann: 3 Stimmen
\item Felix Pielok: 3 Stimmen
\item Thomas G{\"u}tt: 2 Stimmen
\end{itemize}
    Thomas G{\"u}tt wird aus der Wahl f{\"u}r den 2. Beisitzerposten entfernt\\\\
2. Beisitzerposten von 3 - 2. Wahlgang
\begin{itemize}
\item Johannes Schneemann: 5 Stimmen
\item Felix Pielok: 3 Stimmen
\end{itemize}
    Johannes Schneemann ist damit als Beisitzer (Posten 2) gew{\"a}hlt.\\\\
3. Beisitzerposten von 3 - 1. Wahlgang
\begin{itemize}
\item Felix Pielok: 5 Stimmen
\item Thomas G{\"u}tt: 3 Stimmen
\end{itemize}
    Felix Pielok ist damit als Beisitzer (Posten 3) gew{\"a}hlt.\\\\
\\
Die Beisitzer nehmen die Wahl an.

\subsection*{7) Personalien des Vorstandes}
\begin{itemize}
\item Vorstandsvorsitzender:
    Name: Manuel Madrenes\\
	Geburtsdatum: 16.05.1981 in Marseille, wohnhaft in 04229 Leipzig
\item Schatzmeister:
    Name: Steven Chamberlain\\
	Geburtsdatum: 11.04.1987 in Guisborough, Gro{\ss}britannien, wohnhaft in 04177 Leipzig
\item Schriftf{\"u}hrer:
    Name: Jan Hollburg\\
    Geburtsdatum: 07.12.1981 in Naumburg/Saale, wohnhaft in 04277 Leipzig
\item Beisitzer 1:
	Name: Georg Mei{\ss}ner\\
	Geburtsdatum: 31.08.1982 in Halle(Saale), wohnhaft in 04207 Leipzig
\item Beisitzer 2:
	Name: Johannes Schneemann\\
	Geburtsdatum: 26.12.1984 in Heiligenstadt, wohnhaft in 04109 Leipzig
\item Beisitzer 3:
	Name: Felix Pielok\\
	Geburtsdatum: 09.12.1989 in Leipzig, wohnhaft in 04229 Leipzig
\end{itemize}

\subsection*{8) Unterschriften}
Alle Gr{\"u}ndungsmitglieder unterschreiben die Satzung.\\
\\
Versammlungsleiter und Schriftf{\"u}hrer best{\"a}tigen dieses Protokoll mit ihrer Unterschrift\\
\\
\\
\\
\\
\\
\\
Leipzig, den 11.06.2017 \ \ \ Versammlungsleiter: Thomas G{\"u}tt\\
\\
\\
\\
\\
\\
\\
Leipzig, den 11.06.2017 \ \ \ Schriftf{\"u}hrer: Jan Hollburg
\end{document}
