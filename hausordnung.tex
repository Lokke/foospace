\documentclass[10pt,a4paper]{article}
\usepackage{ucs}
\usepackage{eurosym}
\usepackage[utf8x]{inputenc}
\usepackage[ngerman]{babel}
\usepackage[margin=3cm]{geometry}
\newcommand{\qt}[1]{\glq\emph{#1}\grq}
\newcommand{\qs}[1]{\glqq#1\grqq}
\newcommand{\name}{dezentrale}
\newcommand{\revision}{$Revision: 2017-06-11$}
\setlength{\parskip}{6pt}
\setlength{\parindent}{0pt}
\usepackage{enumerate}
\usepackage{color}
\pagestyle{plain}
\usepackage{palatino}
\usepackage[bookmarks,bookmarksopen=true,bookmarksnumbered=true,colorlinks,linkcolor=black,urlcolor=blue]{hyperref}
\usepackage{ulem}
\begin{document}

{\LARGE Hausordnung des \name\ e.V.}

Fassung vom \today (\revision)

TODO: Code of Conduct (auch f�r onlinegeraffel)

%
% Paragraph 1 ====================================================
%

\subsection*{� 1 Zutritt zu den Vereinsr�umen, �ffentliche Veranstaltungen}
% 8.8
\begin{enumerate}
\item Mitglieder haben, vorbehaltlich eines Hausverbotes, jederzeit Zugang zu
den Vereinsr�umen. Als Grundzustand ist der Zugang zu den Vereinsr�umen
Mitgliedern vorbehalten. Mitglieder k�nnen Nichtmitgliedern den Zugang
gew�hren und haften dann gegen�ber dem Verein f�r alle daraus resultierenden Sch�den.
�ffentliche Veranstaltungen haben abweichend dazu eine \"offene T�re\",
Nichtmitglieder sind eingeladen an der Veranstaltung teilzunehmen.
\item Nichtmitglieder d�rfen sich nur bei Anwesenheit eines Mitglieds in den
Vereinsr�umen aufhalten.
\item Geplante oder angedachte �ffentliche Veranstaltungen sind den Mitgliedern und je nach Situation der �ffentlichkeit rechtzeitig angemessen vorzustellen. Ein Mitglied ist f�r Vorbereitung, Ablauf und Nachbereitung der Veranstaltung verantwortlich. Als Grundzustand ist dies das einladende Mitglied, ein anderes Mitglied kann benannt werden. Dieses Mitglied erh�lt f�r die Dauer dieser Veranstaltung das Hausrecht. Es kann das Hausrecht weiter delegieren.
\item Eine Weitergabe von Schl�sseln oder Vergleichbarem zu den Vereinsr�umen an Nichtmitglieder ist untersagt und kann mit sofortigem Vereinsausschluss sanktioniert werden.
\end{enumerate}

%
% Paragraph 2 ====================================================
%
\subsection*{� 2 Umgang mit Gefahrenquellen}
\begin{enumerate}
\item Alle Mitglieder sind angehalten, auftretende Gefahrenquellen zu sichern und dem Vorstand zu melden oder zu beheben, sowie darauf zu achten, keine
    neuen Gefahrenquellen zu erzeugen.
\end{enumerate}

\end{document}

