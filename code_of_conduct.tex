\documentclass[10pt,a4paper]{article}
\usepackage[utf8]{inputenc}
\usepackage[ngerman]{babel}
\usepackage[T1]{fontenc}
\newcommand{\qt}[1]{\glq\emph{#1}\grq}
\newcommand{\qs}[1]{\glqq#1\grqq}
\newcommand{\name}{dezentrale}
\newcommand{\documentstatus}{Entwurf (DRAFT)}
\begin{document}
\title{Code of Conduct}

{\LARGE Code of Conduct}
\section*{Dokumentenstatus}
\documentstatus

\section*{tl;dr}
Be excellent to each other.

\section*{Pr{\"a}ambel}
Die \qs{\name}-community setzt sich aus einer bunt gemischten Menge von
F{\"a}higkeiten, Personalit{\"a}ten und Erfahrungen zusammen. Es ist genau diese
Vielfalt, welche den N{\"a}hrboden f{\"u}r unser Wachstum, Relevanz und Erfolg bildet.
Wenn du mit anderen Mitgliedern der Community interagierst, m{\"o}chten wir dich
bitten, die nachfolgenden Grunds{\"a}tze und Richtlinien zu befolgen, welche dabei helfen sollen,
unser Miteinander zu leiten und die Zeit im FOOSPACE als eine angenehme und
konstruktive Erfahrung zu gestalten.

\section*{Geltungsbereich}
Wir halten den hier verschriftlichten Anspruch an uns und unsere Gemeinschaft
eigentlich f{\"u}r Selbstverst{\"a}ndlichkeiten eines sinnvollen Miteinanders. Ganz
ungeachtet unseres FOOSPACE'es, und als Solches glauben wir, dass es allen
Beteiligten ein Leichtes ist, sie zu beherzigen. Im konkreten Konfliktfall,
gerade in Hinblick auf unseren eigenen Mediations- und Sanktionsanspruch,
betrachten wir diesen \qs{Code of Conduct} als bindend f{\"u}r das Miteinander in
unseren Vereinsr{\"a}umen sowie ggf. durch den Verein betriebener Infrastruktur
(Mailinglisten, DVCS repositories, wikis etc). Ferner erwarten wir ausdr{\"u}cklich,
dass Menschen, welche sich im expliziten Auftrag/Funktion des FOOSPACE in die
{\"O}ffentlichkeit (Vortr{\"a}ge, Konferenzen, Interviews) begeben, sich an den \qs{Code of
Conduct} halten.

\section{Nimm \qs{gute Absichten} an}
Auch wenn du vollkommen anderer Meinung als jemand anderer bist, solltest du
immer davon ausgehen, dass, wenngleich die vorgeschlagene L{\"o}sung vielleicht
wirklich wenig hilfreich ist, hier jemand in bester Absicht versucht, einen
Beitrag zum Gelingen des FOOSPACE zu leisten.

\section{Sei offen und einladend}
Wir ermutigen dich, als ein Mitglied unserer Community offen und einladend
gegen{\"u}ber anderen Mitgliedern wie auch insbesondere gegen{\"u}ber \qs{Besuchern} zu
sein. Gleichwohl verstehen und respektieren wir auch, dass nicht jeder immer
das Interesse oder die Selbstsicherheit hat, mit anderen Menschen zu
interagieren. Das ist auch v{\"o}llig in Ordnung so. Wir bitten dich, deinen Wunsch
\qs{einfach in Ruhe deine Sachen zu machen} einmal deutlich und h{\"o}flich zu
artikulieren, wenn dem in einer Situation so ist. Das mag beim ersten Mal etwas
seltsam sein, wenn du damit noch nicht so viel {\"U}bung hast, aber erm{\"o}glicht es
anderen Menschen, deine W{\"u}nsche zu respektieren und vermeidet Missverst{\"a}ndnisse.

\section{Sei r{\"u}cksichtsvoll}
Mitglieder unserer Community nehmen R{\"u}cksicht auf Ihre Mitmenschen. Wir sind
umsichtig, wenn wir die Bem{\"u}hungen anderer thematisieren, und behalten im
Hinterkopf, dass Ihre Beitr{\"a}ge oft einzig mit dem Ziel geleistet wurden, der Gemeinschaft etwas Gutes
beizusteuern. Wir achten auf unser Kommunikationsverhalten,
gleich, ob pers{\"o}nlich oder online - und wir sind taktvoll, wenn wir auf andere
Ansichten sto{\ss}en. Wenn uns Menschen im Rahmen des FOOSPACE darum bitten „in Ruhe
gelassen zu werden“, verstehen wir dies nicht als pers{\"o}nlichen Angriff, sondern
respektieren dies als legitimes Interesse.

\section{Sei respektvoll}
Mitglieder unserer Community sind respektvoll. Wir sind respektvoll gegen{\"u}ber
anderen, Ihren Haltungen, Ihren F{\"a}higkeiten, Ihren Beitr{\"a}gen und Bem{\"u}hungen.
Bei einem zu ambitionierten und heterogenen Projekt wie dem FOOSPACE wird es
unweigerlich Menschen geben, mit welchen du verschiedener Meinung bist und/oder
es schwierig findest, zusammenzuarbeiten. Wir akzeptiere dies als Tatsache und
bleiben dessen ungeachtet auch bei Meinungsverschiedenheiten respektvoll. Wir
respektieren Ihre freiwillige Arbeit, welche den FOOSPACE erst m{\"o}glich macht.
Meinungsverschiedenheiten sind keine Ausrede f{\"u}r schlechtes Benehmen oder
pers{\"o}nliche Angriffe. Wir respektieren die Abl{\"a}ufe, welche wir uns als
Gemeinschaft selbst gegeben haben, und arbeiten mit ihnen. Wenn wir eine andere
Meinung haben, sind wir h{\"o}flich, wenn wir sie vortragen.

\section*{F{\"u}r den Fall von Problemen}
Wenngleich dieser \qs{Leitfaden} von allen Beteiligten befolgt werden sollte, so
nehmen wir dennoch zur Kenntniss, dass jeder mal einen \qs{schlechten Tag} hat
oder Details der hier gemachten Vorgaben missversteht. Wenn dies geschieht
steht es dir frei, die betreffende Person gerne auf diesen \qs{Code of Conduct} zu verweisen. In
F{\"a}llen, in denen du nicht von Vorsatz oder blanker Nichtachtung ausgehst (wir
hoffen dies ist der \qs{Normalfall}), ist es oft ratsam, dies privat und pers{\"o}nlich
zu tun. Ganz gleich, ob du dich f{\"u}r eine private oder {\"o}ffentliche erste
Thematisierung entscheidest, so sollte der Inhalt doch selbst den hier
genannten Grunds{\"a}tzen folgen.
Solltest du den Wunsch nach Hilfe bei der Problematisierung des kritisierten
Verhaltens haben oder von einem vors{\"a}tzlichen Versto{\ss} ausgehen, bitten wir dich
dich an den Vorstand zu wenden. Solltest du dem Vorstand in dem Vorfall kein
Vertrauen entgegenbringen, steht es dir frei, auf dem in der Satzung und ggf.
Gesch{\"a}ftsordnung genannten Wegen eine Mitgliederversammlung einzuberufen,
welche sich mit dem Thema befasst.
Sollte der Vorfall {\"u}ber den blo{\ss}en Versto{\ss} gegen diesen \qs{Code of Conduct}
hinausgehen und m{\"o}glicherweise strafbewehrt sein, bitten wir dich ggf.
zus{\"a}tzlich an die entsprechenden Beh{\"o}rden zu wenden, da eine vereinsinterne
Mediation/Sanktionierung keinen Ersatz f{\"u}r einen Rechtsvorgang bieten kann.

\section*{Nachtrag}
Alles in allem: wir sind gut zu einander. Wir tragen zu dieser, unserer,
Gemeinschaft nicht bei, weil wir m{\"u}ssen, sondern weil wir wollen. Wenn wir dies
in Erinnerung behalten, wird der Rest von ganz alleine folgen.

Dieser \qs{Code of Conduct} ist ma{\ss}geblich von den entsprechenden Dokumenten der
python und debian community inspiriert. Wir danken Ihnen f{\"u}r Ihre Vorlagen.
\end{document}
