\documentclass[10pt,a4paper]{article}
\usepackage[utf8]{inputenc}
\usepackage[ngerman]{babel}
\usepackage[T1]{fontenc}
\begin{document}
\title{Code of Conduct}
\section*{tl;dr}
Be exelent to each other.

\section*{Präambel}
Die FOOSPACE community setzt sich aus einer bunt gemischten Menge von
Fähigkeiten, Personalitäten und Erfahrungen zusammen. Es ist genau diese
Vielfalt, welche den Nährboden für unser Wachstum, Relevanz und Erfolg bildet.
Wenn du mit anderen Mitgliedern der Community interagierst, möchten wir dich
bitten, die nachfolgenden Grundsätze und Richtlinien zu befolgen, welche dabei helfen sollen,
unser Miteinander zu leiten und die Zeit im FOOSPACE als eine angenehme und
konstruktive Erfahrung zu gestalten.

\section*{Geltungsbereich}
Wir halten den hier verschriftlichten Anspruch an uns und unsere Gemeinschaft
eigentlich für Selbstverständlichkeiten eines sinnvollen Miteinanders. Ganz
ungeachtet unseres FOOSPACE'es, und als Solches glauben wir, dass es allen
Beteiligten ein Leichtes ist, sie zu beherzigen. Im konkreten Konfliktfall,
gerade in Hinblick auf unseren eigenen Mediations- und Sanktionsanspruch,
betrachten wir diesen 'Code of Conduct' als bindend für das Miteinander in
unseren Vereinsräumen sowie ggf. durch den Verein betriebener Infrastruktur
(Mailinglisten, DVCS repositories, wikis etc). Ferner erwarten wir ausdrücklich,
dass Menschen, welche sich im expliziten Auftrag/Funktion des FOOSPACE in die
Öffentlichkeit (Vorträge, Konferenzen, Interviews) begeben, sich an den 'Code of
Conduct' halten.

\section{Nimm 'gute Absichten' an}
Auch wenn du vollkommen anderer Meinung als jemand anderer bist, solltest du
immer davon ausgehen, dass, wenngleich die vorgeschlagene Lösung vielleicht
wirklich wenig hilfreich ist, hier jemand in bester Absicht versucht, einen
Beitrag zum Gelingen des FOOSPACE zu leisten.

\section{Sei offen und einladend}
Wir ermutigen dich, als ein Mitglied unserer Community offen und einladend
gegenüber anderen Mitgliedern wie auch insbesondere gegenüber „Besuchern“ zu
sein. Gleichwohl verstehen und respektieren wir auch, dass nicht jeder immer
das Interesse oder die Selbstsicherheit hat, mit anderen Menschen zu
interagieren. Das ist auch völlig in Ordnung so. Wir bitten dich, deinen Wunsch
„einfach in Ruhe deine Sachen zu machen“ einmal deutlich und höflich zu
artikulieren, wenn dem in einer Situation so ist. Das mag beim ersten Mal etwas
seltsam sein, wenn du damit noch nicht so viel Übung hast, aber ermöglicht es
anderen Menschen, deine Wünsche zu respektieren und vermeidet Missverständnisse.

\section{Sei rücksichtsvoll}
Mitglieder unserer Community nehmen Rücksicht auf Ihre Mitmenschen. Wir sind
umsichtig, wenn wir die Bemühungen anderer thematisieren, und behalten im
Hinterkopf, dass Ihre Beiträge oft einzig mit dem Ziel geleistet wurden, der Gemeinschaft etwas Gutes
beizusteuern. Wir achten auf unser Kommunikationsverhalten,
gleich, ob persönlich oder online - und wir sind taktvoll, wenn wir auf andere
Ansichten stoßen. Wenn uns Menschen im Rahmen des FOOSPACE darum bitten „in Ruhe
gelassen zu werden“, verstehen wir dies nicht als persönlichen Angriff, sondern
respektieren dies als legitimes Interesse.

\section{Sei respektvoll}
Mitglieder unserer Community sind respektvoll. Wir sind respektvoll gegenüber
anderen, Ihren Haltungen, Ihren Fähigkeiten, Ihren Beiträgen und Bemühungen.
Bei einem zu ambitionierten und heterogenen Projekt wie dem FOOSPACE wird es
unweigerlich Menschen geben, mit welchen du verschiedener Meinung bist und/oder
es schwierig findest, zusammenzuarbeiten. Wir akzeptiere dies als Tatsache und
bleiben dessen ungeachtet auch bei Meinungsverschiedenheiten respektvoll. Wir
respektieren Ihre freiwillige Arbeit, welche den FOOSPACE erst möglich macht.
Meinungsverschiedenheiten sind keine Ausrede für schlechtes Benehmen oder
persönliche Angriffe. Wir respektieren die Abläufe, welche wir uns als
Gemeinschaft selbst gegeben haben, und arbeiten mit ihnen. Wenn wir eine andere
Meinung haben, sind wir höflich, wenn wir sie vortragen.

\section*{Für den Fall von Problemen}
Wenngleich dieser 'Leitfaden' von allen Beteiligten befolgt werden sollte, so
nehmen wir dennoch zur Kenntniss, dass jeder mal einen 'schlechten Tag' hat
oder Details der hier gemachten Vorgaben missversteht. Wenn dies geschieht
steht es dir frei, die betreffende Person gerne auf diesen 'Code of Conduct' zu verweisen. In
Fällen, in denen du nicht von Vorsatz oder blanker Nichtachtung ausgehst (wir
hoffen dies ist der 'Normalfall'), ist es oft ratsam, dies privat und persönlich
zu tun. Ganz gleich, ob du dich für eine private oder öffentliche erste
Thematisierung entscheidest, so sollte der Inhalt doch selbst den hier
genannten Grundsätzen folgen.
Solltest du den Wunsch nach Hilfe bei der Problematisierung des kritisierten
Verhaltens haben oder von einem vorsätzlichen Verstoß ausgehen, bitten wir dich
dich an den Vorstand zu wenden. Solltest du dem Vorstand in dem Vorfall kein
Vertrauen entgegenbringen, steht es dir frei, auf dem in der Satzung und ggf.
Geschäftsordnung genannten Wegen eine Mitgliederversammlung einzuberufen,
welche sich mit dem Thema befasst.
Sollte der Vorfall über den bloßen Verstoß gegen diesen 'Code of Conduct'
hinausgehen und möglicherweise strafbewehrt sein, bitten wir dich ggf.
zusätzlich an die entsprechenden Behörden zu wenden, da eine vereinsinterne
Mediation/Sanktionierung keinen Ersatz für einen Rechtsvorgang bieten kann.

\section*{Nachtrag}
Alles in allem: wir sind gut zu einander. Wir tragen zu dieser, unserer,
Gemeinschaft nicht bei, weil wir müssen, sondern weil wir wollen. Wenn wir dies
in Erinnerung behalten, wird der Rest von ganz alleine folgen.

Dieser 'Code of Conduct' ist maßgeblich von den entsprechenden Dokumenten der
python und debian community inspiriert. Wir danken Ihnen für Ihre Vorlagen.
\end{document}
